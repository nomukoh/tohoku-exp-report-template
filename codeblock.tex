\usepackage{xcolor}
\usepackage{listings}
\usepackage{tcolorbox}
\tcbuselibrary{listings, skins, breakable}

% --- 1. カラー定義 (ライトモード用:コントラスト重視) ---
\definecolor{codebg}{HTML}{F5F5F5}       % 背景色:明るいグレー
\definecolor{codecomment}{HTML}{008000}  % コメント:濃い緑
\definecolor{codekeyword}{HTML}{0000FF}  % 予約語:青
\definecolor{codestring}{HTML}{A31515}   % 文字列:濃い赤
\definecolor{codeident}{HTML}{000000}    % 識別子:黒
\definecolor{codedelimiter}{HTML}{000000} % 括弧:黒
\definecolor{codeop}{HTML}{000000}        % 演算子:黒
\definecolor{codenumber}{HTML}{098658}   % 数字:緑がかった青

% --- 2. 独自言語定義 ---
\lstdefinelanguage{mycode}{
    language=Python,
    % 修正箇所:color{white} を color{black} に変更
    basicstyle={\small\ttfamily\color{black}},
    commentstyle={\color{codecomment}\itshape},
    keywordstyle={\color{codekeyword}\bfseries},
    stringstyle={\color{codestring}},
    identifierstyle={\color{codeident}},
    % literate での色指定
    literate={=}{{\textcolor{codeop}{=}}}1
             {+}{{\textcolor{codeop}{+}}}1
             {-}{{\textcolor{codeop}{-}}}1
             {*}{{\textcolor{codeop}{*}}}1
             {/}{{\textcolor{codeop}{/}}}1
             {(}{{\textcolor{codedelimiter}{(}}}1
             {)}{{\textcolor{codedelimiter}{)}}}1
             {[}{{\textcolor{codedelimiter}{[}}}1
             {]}{{\textcolor{codedelimiter}{]}}}1
             {\{}{{\textcolor{codedelimiter}{\{}}}1
             {\}}{{\textcolor{codedelimiter}{\}}}}1
             {0}{{\textcolor{codenumber}{0}}}1
             {1}{{\textcolor{codenumber}{1}}}1
             {2}{{\textcolor{codenumber}{2}}}1
             {3}{{\textcolor{codenumber}{3}}}1
             {4}{{\textcolor{codenumber}{4}}}1
             {5}{{\textcolor{codenumber}{5}}}1
             {6}{{\textcolor{codenumber}{6}}}1
             {7}{{\textcolor{codenumber}{7}}}1
             {8}{{\textcolor{codenumber}{8}}}1
             {9}{{\textcolor{codenumber}{9}}}1,
    numbers=left,
    numberstyle=\tiny\color{gray},
    stepnumber=1,
    numbersep=10pt,
    breaklines=true,
    showstringspaces=false,
    tabsize=4,
}

% --- 3. 表示用環境の定義 ---
\newtcblisting{codeblock}[2][]{
    listing engine=listings,
    listing only,
    colback=codebg,
    colframe=gray!30,     % 枠を背景より少しだけ濃くして境界を明快に
    sharp corners,
    breakable,
    enhanced,
    boxrule=0.5pt,        % 白背景時は少し枠線があった方が締まります
    boxsep=0pt,
    top=5pt, bottom=5pt,
    left=25pt,
    right=5pt,
    width=\linewidth,
    title={#2},
    fonttitle=\small\sffamily\bfseries,
    colbacktitle=gray!20!black, % タイトル部分は暗い色のまま(目立たせる)
    attach boxed title to top left={yshift=-2mm, xshift=2mm},
    boxed title style={sharp corners, boxrule=0pt},
    listing options={
        language=mycode,
        frame=none,
        xleftmargin=0pt,
        lineskip=0.6pt,
        #1
    }
}

\newtcblisting{codeblock*}[1][]{
    listing engine=listings,
    listing only,
    colback=codebg,
    colframe=gray!30,
    sharp corners,
    breakable,
    enhanced,
    boxrule=0.5pt,
    boxsep=0pt,
    top=5pt, bottom=5pt,
    left=25pt,
    right=5pt,
    width=\linewidth,
    listing options={
        language=mycode,
        frame=none,
        xleftmargin=0pt,
        lineskip=0.6pt,
        #1
    }
}
