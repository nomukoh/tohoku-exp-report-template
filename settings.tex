\setpagelayout{noheadfoot,top=20mm,bottom=23mm,left=20mm,right=20mm}
\setlength{\footskip}{8mm}

% パッケージ類
\usepackage{graphicx}   % 画像を挿入するため
\usepackage{amsmath}    % 数式を使うため
\usepackage{amssymb}    % 数式を使うため
\usepackage{caption}    % 図表のキャプション
\usepackage{subcaption} % サブキャプション
\usepackage{siunitx}    % 数式の設定
\usepackage{float}      % 図の配置設定
\usepackage{booktabs}   % 表のスタイル
\usepackage{multirow}   % 表のスタイル
\usepackage{pdfpages}   % 表紙の挿入
\usepackage{enumitem}   % 箇条書き
\usepackage{titlesec}   % セクションのスタイル
\usepackage{luatexja-fontspec} % フォント設定
\usepackage{xspace}     % スタイルの装飾
\usepackage{makecell}   % 表のセルを結合するため
\usepackage{ascmac}     % 図の枠を作成するため
\usepackage{listings}   % ソースコードの表示
\usepackage{comment}    % コメントの使用
\usepackage{fancyhdr}   % ヘッダー・フッター

\usepackage{hyperref}   % 図表へのリンク

% フォント
\setmainjfont{IPAexMincho} 
\AtBeginDocument{%lst
  \fontsize{10.5pt}{17pt}\selectfont
  
  % 数式の上のスペース
  \setlength{\abovedisplayskip}{12pt plus 3pt minus 3pt}
  \setlength{\abovedisplayshortskip}{6pt plus 3pt}
  
  % 数式の下のスペース
  \setlength{\belowdisplayskip}{12pt plus 3pt minus 3pt}
  \setlength{\belowdisplayshortskip}{12pt plus 3pt minus 3pt}
}

% 樹形図
\usepackage{tikz}
\usetikzlibrary{graphs, graphdrawing, arrows.meta, calc}
\usegdlibrary{trees}  % trees 用ライブラリ

% セクションスタイル
\newcommand{\mysectionstyle}[2]{%
  \sffamily         % 欧文・数字→サンセリフ
  \gtfamily         % 日本語→ゴシック
  \bfseries
  \fontsize{#1}{#2}\selectfont
}

% セクションのスタイル
\titleformat{\section}
  {\mysectionstyle{14pt}{15pt}}
  {\mysectionstyle{14pt}{15pt}\thesection.}
  {0.7em}{}
\titlespacing*{\section}{0em}{1.5em}{0.5em}

% サブセクションのスタイル
\titleformat{\subsection}
  {\mysectionstyle{12pt}{15pt}}
  {\mysectionstyle{12pt}{15pt}\thesubsection.}
  {0.7em}{}
\titlespacing*{\subsection}{0em}{1.5em}{0.5em}

% サブサブセクションのスタイル
\titleformat{\subsubsection}
  {\mysectionstyle{11.5pt}{12pt}}
  {\mysectionstyle{11.5pt}{12pt}\thesubsubsection.}
  {0em}{}
\titlespacing*{\subsubsection}{0em}{1.0em}{0em}

% 和文キャプション
\renewcommand{\figurename}{図}
\renewcommand{\tablename}{表}
\renewcommand{\refname}{参考文献}
\renewcommand{\lstlistingname}{ソースコード}

% autoref の和文化
\renewcommand*{\figureautorefname}{図}
\renewcommand*{\tableautorefname}{表}
\renewcommand*{\equationautorefname}{式}
\renewcommand*{\subfigureautorefname}{図}

\captionsetup[subfigure]{labelformat=simple}
\renewcommand*{\thesubfigure}{(\alph{subfigure})}

\captionsetup[subtable]{labelformat=simple}
\renewcommand*{\thesubtable}{(\alph{subtable})}

% ブラケット付き単位の設定
\NewDocumentCommand{\sibr}{m}{\,\text{[\si{#1}]}}

% ソースコード用の設定
\lstset{
  basicstyle={\ttfamily},
  identifierstyle={\small},
  commentstyle={\small\textnormal},
  keywordstyle={\small\bfseries},
  keepspaces=true,
  ndkeywordstyle={\small},
  stringstyle={\small\ttfamily},
  frame={tb},
  breaklines=true,
  columns=[l]{fullflexible},
  %columns=fixed, % 文字幅を合わせる
  numbers=left,
  xrightmargin=0em,
  xleftmargin=3em,
  numbersep=1em,
  numberstyle={\scriptsize},
  stepnumber=1,
  lineskip=-0.5ex
}

% リンクの色設定
\hypersetup{
  colorlinks=true,
  linkcolor=black,
  citecolor=black,
  urlcolor=black
}

% 参考文献の設定
\usepackage[backend=biber, style=numeric, sorting=none]{biblatex}
\addbibresource{refs.bib}

% --- 基本的な区切り文字と書式の設定 ---
\renewcommand*{\newunitpunct}{,}       % 各項目の区切りを「,」に
\renewcommand*{\finentrypunct}{.}      % 末尾を「.」に
\DeclareDelimFormat{multinamedelim}{,} % 著者名間の区切り
\DeclareDelimFormat{finalnamedelim}{,} % 最後の著者名の区切り

% --- フィールドごとの書式設定 ---

% タイトル:論文・章・Webは二重引用符,書籍・雑誌はそのまま
\DeclareFieldFormat{title}{#1} 
\DeclareFieldFormat[article,inbook,incollection,online]{title}{“#1,”} 
\DeclareFieldFormat{journaltitle}{#1}
\DeclareFieldFormat{booktitle}{#1}

% ページ番号:pp. を付ける
\DeclareFieldFormat{pages}{pp.#1}

% URLと参照日
\DeclareFieldFormat{url}{#1}
\DeclareFieldFormat{urldate}{,\thefield{urlyear}/\thefield{urlmonth}/\thefield{urlday}}

% --- 引用箇所の表示([1]の上付き) ---
% ガイドには「[1]」とありますが,本文中で上付きにしたい場合はこのまま残してください.
% 普通の [1] にしたい場合は,以下の\DeclareCiteCommandブロックを削除してください.
\DeclareCiteCommand{\cite}
  {\usebibmacro{prenote}}
  {\mkbibsuperscript{[\printfield{labelnumber}]}}
  {\multicitedelim}
  {\usebibmacro{postnote}}

% --- ドライバ(表示順序)の設定 ---

% [1] 論文誌 (article)
% 著者名,“標題,” 雑誌名,巻,号,pp.始め-終り,月年.
\DeclareBibliographyDriver{article}{%
  \printnames{author}%
  \newunit\newblock
  \printfield{title}% 引用符とカンマはFieldFormatで処理
  \printfield{journaltitle}%
  \newunit\newblock
  \printfield{volume}%
  \newunit\newblock
  \printfield{number}%
  \newunit\newblock
  \printfield{pages}%
  \newunit\newblock
  % 日付(月・年)の処理
  \printdate
  \finentry
}

% [2] 著書・編書 (book)
% 著者名,書名,編者名,発行所,発行年.
% \DeclareBibliographyDriver{book}{%
%   \printnames{author}%
%   \newunit\newblock
%   \printfield{title}%
%   \newunit\newblock
%   \printnames{editor}%
%   \newunit\newblock
%   \printlist{publisher}%
%   \newunit\newblock
%   \printfield{year}%
%   \finentry
% }

% [3] 著書の一部 (book)
% 著者名,“標題,” 書名,編者名,章番号またはpp.,発行所,発行年.
\DeclareBibliographyDriver{book}{%
  \printnames{author}%
  \newunit\newblock
  \printfield{title}%
  \printfield{booktitle}%
  \newunit\newblock
  \printnames{editor}%
  \newunit\newblock
  \iffieldundef{chapter}
    {\printfield{pages}}
    {\printfield{chapter}}%
  \newunit\newblock
  \printlist{publisher}%
  \newunit\newblock
  \printfield{year}%
  \finentry
}

% [4] Webページ (online)
% “Web ページのタイトル”,URL アドレス,アクセスした日付.
\DeclareBibliographyDriver{online}{%
  \printfield{title}%
  \newunit\newblock
  \printfield{url}%
  % \newunitを入れるとカンマが入るが,urldateのフォーマット内にカンマを含めているためここは不要
  \printurldate
  \finentry
}

% 段落設定
\setlength{\parskip}{0em}
\setlength{\parindent}{1em}
\makeatletter
\AtBeginDocument{
  \let\@afterindentfalse\@afterindenttrue
}
\makeatother

% ヘッダーとフッター設定
\pagestyle{fancy}  % ページスタイルを変更
\fancyhf{}  % デフォルトのヘッダー・フッターをクリア
\renewcommand{\headrulewidth}{0pt} % ヘッダーの横線を削除
\fancyfoot[C]{\thepage}  % ページ番号を中央下に配置
